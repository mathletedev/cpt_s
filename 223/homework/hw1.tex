\documentclass[12pt]{article}
\usepackage[fleqn]{amsmath}
\usepackage{amsthm,enumitem,minted}
\newtheorem{thm}{Theorem}

\begin{document}

\begin{center}
  {\Large CPT\_S 223 Homework 1}
  $ $\\
  $ $\\
  \begin{tabular}{rl}
    WSU ID: & 11870028 \\
    Name: & Neal Wang \\
    Due Date: & 16 February 2025
  \end{tabular}
\end{center}

\section*{Problem 1}

\begin{thm}
  If $N(k)$ is the number of nodes of a complete binary tree at level
  $k$, then $N(k) = 2^k$.
\end{thm}

\begin{proof}
  $ $\\
  Base case: If $k = 0$, then $N(0) = 1$. \\
  Inductive case: Each node has exactly two children
  \begin{align*}
    N(k) & = 2N(k - 1) \\
    & = 2 \cdot 2N(k - 2) \\
    & = 2^mN(k - m) \\
    & = 2^kN(k - k) \\
    & = 2^kN(0) \\
    & = 2^k \cdot 1 \\
    & = 2^k
  \end{align*}
  Therefore, $N(k) = 2^k$.
\end{proof}

\section*{Problem 2}

\begin{minted}{cpp}
int height(Node *p_node, int h = 0) {
  if (p_node == nullptr) {
    return h;
  }

  return std::max(height(p_node->left, h + 1), height(p_node->right, h + 1));
}
\end{minted}

\section*{Problem 3}

\section*{Problem 4}

\begin{enumerate}[label=(\alph*)]
  \item (your solution)
  \item (your solution)
\end{enumerate}

\end{document}
