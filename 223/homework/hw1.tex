\documentclass[12pt]{article}
\usepackage[fleqn]{amsmath}
\usepackage{amsthm,amsfonts,amssymb,braket,cleveref,enumitem,minted}

\newtheorem{theorem}{Theorem}[section]
\newtheorem{corollary}{Corollary}[theorem]
\newtheorem{lemma}[theorem]{Lemma}
\theoremstyle{definition}
\newtheorem{definition}{Definition}[section]
\newenvironment{solution}
{\renewcommand\qedsymbol{$\blacksquare$}
\begin{proof}[Solution]}
  {
\end{proof}}

\begin{document}

\begin{center}
  {\Large CPT\_S 223 Homework 1}
  $ $\\
  $ $\\
  \begin{tabular}{rl}
    WSU ID: & 11870028 \\
    Name: & Neal Wang \\
    Due Date: & 16 February 2025
  \end{tabular}
\end{center}

\section{Problem 1}

\begin{theorem}
  Let $N(k)$ denote the number of nodes of a complete binary tree at level
  $k$. Then $N(k) = 2^k$.
\end{theorem}

\begin{proof}
  Base case: If $k = 0$, then $N(0) = 1$.

  Inductive case: Each node has exactly two children
  \begin{align*}
    N(k) & = 2N(k - 1) \\
    & = 2 \cdot 2N(k - 2) \\
    & = 2^mN(k - m) \\
    & = 2^kN(k - k) \\
    & = 2^kN(0) \\
    & = 2^k \cdot 1 \\
    & = 2^k
  \end{align*}

  Therefore, $N(k) = 2^k$.
\end{proof}

\section{Problem 2}

\begin{minted}{cpp}
int height(Node *p_node, int h = 0) {
  // base case: no node
  if (p_node == nullptr) {
    return h;
  }

  // return the greater height between the left and right subtrees
  return std::max(height(p_node->left, h + 1), height(p_node->right, h + 1));
}
\end{minted}

\section{Problem 3}

\[
  T(n) =
  \begin{cases}
    3, & n = 1 \\
    3T(n / 3) + n, & n > 1 \land n \in \set{n \mid n = 3^k, k \in \mathbb N}
  \end{cases}
\]

\begin{solution}
  \begin{align*}
    T(n) & = 3T\left(\dfrac{n}{3}\right) + n \\
    & = 3 \cdot \left(3T\left(\dfrac{n}{3 \cdot 3}\right) + n / 3\right) + n \\
    & = 3 \cdot 3T\left(\dfrac{n}{3 \cdot 3}\right) + n + n \\
    & = 3^kT\left(\dfrac{n}{3^k}\right) + kn \\
    & = 3^{\log_3 n}T\left(\dfrac{n}{3^{\log_3 n}}\right) + \log_3 n \cdot n \\
    & = nT\left(\dfrac{n}{n}\right) + n \log_3 n \\
    & = nT(1) + n \log_3 n \\
    & = 3n + n \log_3 n \\
  \end{align*}

  Therefore, $T(n) = n \log_3 n + 3n$.
\end{solution}

\section{Problem 4}

\begin{enumerate}[label=(\alph*)]
  \item
    \begin{enumerate}[label=(\roman*)]
      \item Lambda expressions and \verb|<functional>|: Introduced
        functional programming paradigm.
      \item \verb|nullptr|: A strongly-typed null pointer that
        replaced the bug-prone \verb|NULL| macro.
    \end{enumerate}
  \item How C++ handles null pointers:
    \begin{minted}{cpp}
int square(int x) { return x * x; }
square(NULL); // bad: implicit conversion, may cause ub
square(nullptr); // good: causes compile error
    \end{minted}
  \item C++ 14 introduced generic lambdas, which allow for the use of
    \verb|auto| in lambda function parameters.
\end{enumerate}

\section{Problem 5}

\begin{definition}[Clockwise-adjacent]
  A peg $I$ is clockwise-adjacent to a peg $J$ if $I$ is reachable from
  $J$ in 1 clockwise move.
\end{definition}

\begin{definition}[Counterclockwise-adjacent]
  A peg $I$ is counterclockwise-adjacent to a peg $J$ if $I$ is reachable from
  $J$ in 2 clockwise moves.
\end{definition}

\begin{lemma}
  \label{basecases}
  For $Q(1) = 1$, we can transfer a single disk from $A \to B$ using 1
  clockwise move.

  Similarly, for $R(1) = 2$ we can transfer a single disk
  from $A \to C$ using 2 clockwise moves.
\end{lemma}

\begin{lemma}
  \label{minmoves}
  The minimum number of moves required to transfer $n$ disks from $A
  \to B$ using $C$ is equivalent to the minimum number of moves
  required to transfer $n$ disks from $B \to C$ using $A$.
\end{lemma}

\begin{proof}
  This can be shown by simply renaming the pegs by one clockwise
  rotation. I.e. $A = B, B = C, C = A$.
\end{proof}

And a consequence of \cref{minmoves} is the statement in the
next corollary.

\begin{corollary}
  \label{genfuncs}
  $Q(n)$ is equivalent to the minimum number of moves required to transfer $n$
  disks from any peg to its clockwise-adjacent peg.

  Similarly, $R(n)$ is equivalent to the minimum number of moves required to
  transfer $n$ disks from any peg to its counterclockwise-adjacent
  peg.
\end{corollary}

\begin{lemma}
  \label{q}
  If $n > 1$, then $Q(n) = 2R(n - 1) + 1$.
\end{lemma}

\begin{proof}
  First, we transfer $n - 1$ disks to their
  counterclockwise-adjacent peg. By \cref{genfuncs}, this requires
  $R(n - 1)$ moves.

  Next, we transfer the largest disk to its clockwise-adjacent peg. By
  \cref{basecases}, this requires 1 move.

  Finally, we transfer the $n - 1$ disks to their
  counterclockwise-adjacent peg. By \cref{genfuncs}, this requires
  $R(n - 1)$ moves.

  In total, we used $R(n - 1) + 1 + R(n - 1) = 2R(n - 1) + 1$ moves.

  Because no empty peg is unused, we can conclude that this is the
  minimum number of moves required.

  Therefore, $Q(n) = 2R(n - 1) + 1$.
\end{proof}

\begin{lemma}
  \label{r}
  If $n > 1$, then $R(n) = Q(n) + Q(n - 1) + 1$.
\end{lemma}

\begin{proof}
  We observe that the proof for \cref{q} will not work for $R$.
  This is because we cannot transfer the largest disk across a stack
  of smaller disks. Thus, we will use an alternative approach.

  First, we transfer $n - 1$ disks to their
  counterclockwise-adjacent peg. By \cref{genfuncs}, this
  requires $R(n - 1)$ moves. This creates a gap to transfer the largest disk.

  Next, we transfer the largest disk to its clockwise-adjacent
  peg. By \cref{basecases}, this requires 1 move.

  Next, we transfer the $n - 1$ disks to their
  clockwise-adjacent peg. By \cref{genfuncs}, this requires
  $Q(n - 1)$ moves. This creates another gap to transfer the largest disk.

  Next, we transfer the largest disk to its clockwise-adjacent
  peg. By \cref{basecases}, this requires 1 move.

  Finally, we transfer the $n - 1$ disks to their
  counterclockwise-adjacent peg. By \cref{genfuncs}, this
  requires $R(n - 1)$ moves.

  In total, we used $R(n - 1) + 1 + Q(n - 1) + 1 + R(n - 1)$ moves.

  We observe that $Q(n) = 2R(n - 1) + 1$, so we can rewrite this as
  $Q(n) + Q(n - 1) + 1$.

  Therefore, $R(n) = Q(n) + Q(n - 1) + 1$.
\end{proof}

\begin{theorem}[Tower of Hanoi]
  \leavevmode
  \begin{enumerate}[label=(\arabic*)]
    \item \[
        Q(n) =
        \begin{cases}
          1, & n = 1 \\
          2R(n - 1) + 1, & n > 1
        \end{cases}
      \]
    \item \[
        R(n) =
        \begin{cases}
          2, & n = 1 \\
          Q(n) + Q(n - 1) + 1, & n > 1
        \end{cases}
      \]
  \end{enumerate}
\end{theorem}

\begin{proof}
  For $n = 1$, we apply \cref{basecases}, which gives $Q(1) = 1$ and $R(1) = 2$.

  For $n > 1$, we apply \cref{q,r}, which gives $Q(n) = 2R(n - 1) +
  1$ and $R(n) = Q(n) + Q(n - 1) + 1$.

  Therefore, the piecewise definitions hold, completing the proof.
\end{proof}

\end{document}
