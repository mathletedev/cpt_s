\documentclass[12pt]{article}
\usepackage[fleqn]{amsmath}
\usepackage{amsthm,amsfonts,amssymb,braket,enumitem,minted,hyperref,cleveref}

\newtheorem{theorem}{Theorem}[section]
\newtheorem{corollary}{Corollary}[theorem]
\newtheorem{lemma}[theorem]{Lemma}
\theoremstyle{definition}
\newtheorem{definition}{Definition}[section]
\newenvironment{solution}
{\renewcommand\qedsymbol{$\blacksquare$}
\begin{proof}[Solution]}
  {
\end{proof}}
\hypersetup{
  colorlinks=true,
  urlcolor=violet,
}

\begin{document}

\begin{center}
  {\Large CPT\_S 223 Homework 2}
  $ $\\
  $ $\\
  \begin{tabular}{rl}
    WSU ID: & 11870028 \\
    Name: & Neal Wang \\
    Due Date: & 2 March 2025
  \end{tabular}
\end{center}

\section{Problem 1}

\begin{minted}{c}
node *reverse(node *head) {
  node *prev = NULL;
  node *curr = head;

  while (curr != NULL) {
    node *next = curr->next;
    curr->next = prev;
    prev = curr;
    curr = next;
  }

  return prev;
}
\end{minted}

\newpage

\section{Problem 2}

This is an application of
\href{https://en.wikipedia.org/wiki/Cycle_detection#Floyd's_tortoise_and_hare}
{Floyd's tortoise and hare algorithm}. I've previously implemented
this algorithm in C++ for a Codeforces contest:
\href{https://github.com/mathletedev/cp/blob/main/Codeforces/1137D.cpp}{GitHub}

Here is an algorithmic pseudocode for cycle detection:

\begin{minted}{c}
// to avoid null dereference
node *next(node *curr) {
  return curr == NULL ? NULL : curr->next;
}

bool cyclic(node *head) {
  node *hare = head, *tortoise = head;

  while (NULL) {
    hare = next(next(hare));
    tortoise = next(tortoise);

    // list terminates, thus is acyclic
    if (hare == NULL) {
      return false;
    }

    // hare and tortoise meet, thus is cyclic
    if (hare == tortoise) {
      return true;
    }
  }
}
\end{minted}

Since the hare travels at twice the speed of the tortoise, it
advances one extra node per iteration. Once the hare leaves the
starting node, the only way for the hare to encounter the tortoise
again is if there is a cycle in the linked list.

\newpage

\section{Problem 3}

(a), (b), (d), and (f) all apply.

\section{Problem 4}

$37, 2 / N, \sqrt N, N, N \log(\log N), \set{N \log N, N \log N^2}, N \log^2
N, \\ N^{1.5}, N^2, N^2 \log N, N^3, 2^{N / 2}, 2^N$

\section{Problem 5}

\subsection{}

Prove that $5 \log N = o(N \log N)$.

\subsection{}

Prove that $2 \log N = O(\sqrt N)$.

\begin{lemma}
  \label{growth}
  $\log N \ll \sqrt N$ ($\sqrt N$ grows faster than $\log N$).
\end{lemma}

\begin{proof}
  We will analyse $\lim_{N \to \infty} \dfrac{\log N}{\sqrt N}$.

  \begin{align*}
    \lim_{N \to \infty} \dfrac{\log N}{\sqrt N} & =
    \lim_{N \to \infty} \dfrac{\frac{d}{dN} \log N}{\frac{d}{dN}
    \sqrt N} \quad \text{(By
    L'Hôpital's Rule)} \\
    & = \lim_{N \to \infty} \dfrac{\frac{1}{N}}{\frac{1}{2 \sqrt N}} \\
    & = \lim_{N \to \infty} \dfrac{2 \sqrt N}{N} \\
    & = \lim_{N \to \infty} \dfrac{1}{\sqrt N} \\
    & = 0
  \end{align*}

  Therefore, $\log N \ll \sqrt N$.
\end{proof}

\end{document}
