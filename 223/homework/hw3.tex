\documentclass[12pt]{article}
\usepackage[fleqn]{amsmath}
\usepackage{amsthm,amsfonts,amssymb,braket,enumitem,minted,hyperref,cleveref}

\newtheorem{theorem}{Theorem}[section]
\newtheorem{corollary}{Corollary}[theorem]
\newtheorem{lemma}[theorem]{Lemma}
\theoremstyle{definition}
\newtheorem{definition}{Definition}[section]
\newenvironment{solution}
{\renewcommand\qedsymbol{$\blacksquare$}
\begin{proof}[Solution]}
  {
\end{proof}}
\hypersetup{
  colorlinks=true,
  urlcolor=violet,
  linkcolor=blue,
}

\begin{document}

\begin{center}
  {\Large CPT\_S 223 Homework 3}
  $ $\\
  $ $\\
  \begin{tabular}{rl}
    WSU ID: & 11870028 \\
    Name: & Neal Wang \\
    Due Date: & 27 March 2025
  \end{tabular}
\end{center}

\section{Problem 1}

\begin{enumerate}[label=(\alph*)]
  \item 13
  \item 7
  \item 5
  \item 3
  \item 2
  \item Post-order: G H F K L M J I B D E C A \\
    Pre-order: A B F G H I K J L M C D E
  \item Level-order: A B C F I D E G H K J L M \\
    In-order: G F H B K I L J M A D C E
\end{enumerate}

\section{Problem 2}

\section{Problem 3}

A tree with $n$ nodes has exactly $n - 1$ edges. This is because each
node has exactly one edge to its parent, but the root node has no parent.

Another way to think about this is that a tree is an acyclic
connected graph. There must be at least $n - 1$ edges in a connected
graph. There also can't be more than $n - 1$ edges, since this would
result in a cyclic graph due to the Pigeonhole Principle.

\end{document}
