\documentclass[12pt]{article}
\usepackage[fleqn]{amsmath}
\usepackage{amsthm,amsfonts,amssymb,braket,enumitem,minted,hyperref,cleveref,tikz}

\newtheorem{theorem}{Theorem}[section]
\newtheorem{corollary}{Corollary}[theorem]
\newtheorem{lemma}[theorem]{Lemma}
\theoremstyle{definition}
\newtheorem{definition}{Definition}[section]
\newenvironment{solution}
{\renewcommand\qedsymbol{$\blacksquare$}
\begin{proof}[Solution]}
  {
\end{proof}}
\hypersetup{
  colorlinks=true,
  urlcolor=violet,
  linkcolor=blue,
}

\begin{document}

\begin{center}
  {\Large CPT\_S 223 Homework 3}
  $ $\\
  $ $\\
  \begin{tabular}{rl}
    WSU ID: & 11870028 \\
    Name: & Neal Wang \\
    Due Date: & 27 March 2025
  \end{tabular}
\end{center}

\section{Problem 1}

\begin{enumerate}[label=(\alph*)]
  \item 13
  \item 7
  \item 5
  \item 3
  \item 2
  \item Post-order: G H F K L M J I B D E C A \\
    Pre-order: A B F G H I K J L M C D E
  \item Level-order: A B C F I D E G H K J L M \\
    In-order: G F H B K I L J M A D C E
\end{enumerate}

\section{Problem 2}

\begin{tikzpicture}[level distance=1.5cm,
    level 1/.style={sibling distance=3cm},
  level 2/.style={sibling distance=1.5cm}]
  \node{C}
  child{node{B}
    child{node{A}}
    child{node{I}
      child{node{K}
        child{node{L}}
        child{node{M}}
      }
      child{node{J}}
    }
  }
  child{node{D}
    child{node{H}}
    child{node{E}
      child{node{G}}
      child{node{F}}
    }
  };
\end{tikzpicture}

\section{Problem 3}

A tree with $n$ nodes has exactly $n - 1$ edges. This is because each
node has exactly one edge to its parent, but the root node has no parent.

Another way to think about this is that a tree is an acyclic
connected graph. There must be at least $n - 1$ edges in a connected
graph. There also can't be more than $n - 1$ edges, since this would
result in a cyclic graph due to the Pigeonhole Principle.

\section{Problem 4}

\begin{enumerate}[label=(\alph*)]
  \item
    \begin{enumerate}[label=(\alph*)]
      \item
        \begin{tikzpicture}[level distance=1cm,
          baseline={([yshift=-1em] current bounding box.north)}]
          \node{8}
          child{node{7}
            child{node{1}
              child{node{}}
              child{node{6}
                child{node{2}
                  child{node{}}
                  child{node{3}
                    child{node{}}
                    child{node{5}
                      child{node{4}}
                      child{node{}}
                    }
                  }
                }
                child{node{}}
              }
            }
            child{node{}}
          }
          child{node{}};
        \end{tikzpicture}
      \item
        \begin{tikzpicture}[level distance=1cm,
            level 1/.style={sibling distance=2cm},
            level 2/.style={sibling distance=1cm},
          baseline={([yshift=-1em] current bounding box.north)}]
          \node{5}
          child{node{3}
            child{node{2}
              child{node{1}}
              child{node{}}
            }
            child{node{4}}
          }
          child{node{7}
            child{node{6}}
            child{node{8}}
          };
        \end{tikzpicture}
    \end{enumerate}
  \item The first tree is much taller than the second tree; it's a
    degenerate binary tree, essentially a linked list. The second
    tree is much more balanced, meaning operations would be more
    efficient on it.
\end{enumerate}

\section{Problem 5}

\subsection{(a)}

\begin{enumerate}[label=(\arabic*)]
  \item
    \begin{tikzpicture}[level distance=1cm,
        level 1/.style={sibling distance=2cm},
        level 2/.style={sibling distance=1cm},
      baseline={([yshift=-1em] current bounding box.north)}]
      \node{7};
    \end{tikzpicture}
  \item
    \begin{tikzpicture}[level distance=1cm,
        level 1/.style={sibling distance=2cm},
        level 2/.style={sibling distance=1cm},
      baseline={([yshift=-1em] current bounding box.north)}]
      \node{7}
      child{node{5}}
      child{node{}};
    \end{tikzpicture}
  \item
    \begin{tikzpicture}[level distance=1cm,
        level 1/.style={sibling distance=2cm},
        level 2/.style={sibling distance=1cm},
      baseline={([yshift=-1em] current bounding box.north)}]
      \node{7}
      child{node{5}
        child{node{2}}
        child{node{}}
      }
      child{node{}};
    \end{tikzpicture}
    Node 7 is unbalanced, LL case
    \begin{tikzpicture}[level distance=1cm,
        level 1/.style={sibling distance=2cm},
        level 2/.style={sibling distance=1cm},
      baseline={([yshift=-1em] current bounding box.north)}]
      \node{5}
      child{node{2}}
      child{node{7}};
    \end{tikzpicture}
  \item
    \begin{tikzpicture}[level distance=1cm,
        level 1/.style={sibling distance=2cm},
        level 2/.style={sibling distance=1cm},
      baseline={([yshift=-1em] current bounding box.north)}]
      \node{5}
      child{node{2}
        child{node{}}
        child{node{4}}
      }
      child{node{7}};
    \end{tikzpicture}
  \item
    \begin{tikzpicture}[level distance=1cm,
        level 1/.style={sibling distance=2cm},
        level 2/.style={sibling distance=1cm},
      baseline={([yshift=-1em] current bounding box.north)}]
      \node{5}
      child{node{2}
        child{node{}}
        child{node{4}
          child{node{3}}
          child{node{}}
        }
      }
      child{node{7}};
    \end{tikzpicture}
    Node 2 is unbalanced, RL case
    \begin{tikzpicture}[level distance=1cm,
        level 1/.style={sibling distance=2cm},
        level 2/.style={sibling distance=1cm},
      baseline={([yshift=-1em] current bounding box.north)}]
      \node{5}
      child{node{3}
        child{node{2}}
        child{node{4}}
      }
      child{node{7}};
    \end{tikzpicture}
  \item
    \begin{tikzpicture}[level distance=1cm,
        level 1/.style={sibling distance=2cm},
        level 2/.style={sibling distance=1cm},
      baseline={([yshift=-1em] current bounding box.north)}]
      \node{5}
      child{node{3}
        child{node{2}
          child{node{1}}
          child{node{}}
        }
        child{node{4}}
      }
      child{node{7}};
    \end{tikzpicture}
    Node 5 is unbalanced, LL case
    \begin{tikzpicture}[level distance=1cm,
        level 1/.style={sibling distance=2cm},
        level 2/.style={sibling distance=1cm},
      baseline={([yshift=-1em] current bounding box.north)}]
      \node{3}
      child{node{2}
        child{node{1}}
        child{node{}}
      }
      child{node{5}
        child{node{4}}
        child{node{7}}
      };
    \end{tikzpicture}
\end{enumerate}

\subsection{(b)}

\begin{enumerate}[label=(\arabic*)]
  \item
    \begin{tikzpicture}[level distance=1cm,
        level 1/.style={sibling distance=2cm},
        level 2/.style={sibling distance=1cm},
      baseline={([yshift=-1em] current bounding box.north)}]
      \node{2};
    \end{tikzpicture}
  \item
    \begin{tikzpicture}[level distance=1cm,
        level 1/.style={sibling distance=2cm},
        level 2/.style={sibling distance=1cm},
      baseline={([yshift=-1em] current bounding box.north)}]
      \node{2}
      child{node{1}}
      child{node{}};
    \end{tikzpicture}
  \item
    \begin{tikzpicture}[level distance=1cm,
        level 1/.style={sibling distance=2cm},
        level 2/.style={sibling distance=1cm},
      baseline={([yshift=-1em] current bounding box.north)}]
      \node{2}
      child{node{1}}
      child{node{4}};
    \end{tikzpicture}
  \item
    \begin{tikzpicture}[level distance=1cm,
        level 1/.style={sibling distance=2cm},
        level 2/.style={sibling distance=1cm},
      baseline={([yshift=-1em] current bounding box.north)}]
      \node{2}
      child{node{1}}
      child{node{4}
        child{node{}}
        child{node{5}}
      };
    \end{tikzpicture}
  \item
    \begin{tikzpicture}[level distance=1cm,
        level 1/.style={sibling distance=2cm},
        level 2/.style={sibling distance=1cm},
      baseline={([yshift=-1em] current bounding box.north)}]
      \node{2}
      child{node{1}}
      child{node{4}
        child{node{}}
        child{node{5}
          child{node{}}
          child{node{9}}
        }
      };
    \end{tikzpicture}
    Node 4 is unbalanced, RR case
    \begin{tikzpicture}[level distance=1cm,
        level 1/.style={sibling distance=2cm},
        level 2/.style={sibling distance=1cm},
      baseline={([yshift=-1em] current bounding box.north)}]
      \node{2}
      child{node{1}}
      child{node{5}
        child{node{4}}
        child{node{9}}
      };
    \end{tikzpicture}
  \item
    \begin{tikzpicture}[level distance=1cm,
        level 1/.style={sibling distance=2cm},
        level 2/.style={sibling distance=1cm},
      baseline={([yshift=-1em] current bounding box.north)}]
      \node{2}
      child{node{1}}
      child{node{5}
        child{node{4}
          child{node{3}}
          child{node{}}
        }
        child{node{9}}
      };
    \end{tikzpicture}
    Node 2 is unbalanced, RL case
    \begin{tikzpicture}[level distance=1cm,
        level 1/.style={sibling distance=2cm},
        level 2/.style={sibling distance=1cm},
      baseline={([yshift=-1em] current bounding box.north)}]
      \node{4}
      child{node{2}
        child{node{1}}
        child{node{3}}
      }
      child{node{5}
        child{node{}}
        child{node{9}}
      };
    \end{tikzpicture}
  \item
    \begin{tikzpicture}[level distance=1cm,
        level 1/.style={sibling distance=2cm},
        level 2/.style={sibling distance=1cm},
      baseline={([yshift=-1em] current bounding box.north)}]
      \node{4}
      child{node{2}
        child{node{1}}
        child{node{3}}
      }
      child{node{5}
        child{node{}}
        child{node{9}
          child{node{6}}
          child{node{}}
        }
      };
    \end{tikzpicture}
    Node 5 is unbalanced, RL case
    \begin{tikzpicture}[level distance=1cm,
        level 1/.style={sibling distance=2cm},
        level 2/.style={sibling distance=1cm},
      baseline={([yshift=-1em] current bounding box.north)}]
      \node{4}
      child{node{2}
        child{node{1}}
        child{node{3}}
      }
      child{node{6}
        child{node{5}}
        child{node{9}}
      };
    \end{tikzpicture}
  \item
    \begin{tikzpicture}[level distance=1cm,
        level 1/.style={sibling distance=2cm},
        level 2/.style={sibling distance=1cm},
      baseline={([yshift=-1em] current bounding box.north)}]
      \node{4}
      child{node{2}
        child{node{1}}
        child{node{3}}
      }
      child{node{6}
        child{node{5}}
        child{node{9}
          child{node{7}}
          child{node{}}
        }
      };
    \end{tikzpicture}
\end{enumerate}

\end{document}
