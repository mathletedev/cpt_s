\documentclass[12pt]{article}
\usepackage[fleqn]{amsmath}
\usepackage{amsthm,amsfonts,amssymb,braket,enumitem,minted,hyperref,cleveref}

\newtheorem{theorem}{Theorem}[section]
\newtheorem{corollary}{Corollary}[theorem]
\newtheorem{lemma}[theorem]{Lemma}
\theoremstyle{definition}
\newtheorem{definition}{Definition}[section]
\newenvironment{solution}
{\renewcommand\qedsymbol{$\blacksquare$}
\begin{proof}[Solution]}
  {
\end{proof}}
\hypersetup{
  colorlinks=true,
  urlcolor=violet,
  linkcolor=blue,
}

\begin{document}

\begin{center}
  {\Large CPT\_S 223 Homework 2}
  $ $\\
  $ $\\
  \begin{tabular}{rl}
    WSU ID: & 11870028 \\
    Name: & Neal Wang \\
    Due Date: & 2 March 2025
  \end{tabular}
\end{center}

\section{Problem 1}

\begin{minted}{c}
node *reverse(node *head) {
  // keep track of previous node to replace next with
  node *prev = NULL;
  node *curr = head;

  while (curr != NULL) {
    // reverse link
    node *next = curr->next;
    curr->next = prev;

    // move to next node
    prev = curr;
    curr = next;
  }

  // return new head (aka last node)
  return prev;
}
\end{minted}

\newpage

\section{Problem 2}

This is an application of
\href{https://en.wikipedia.org/wiki/Cycle_detection#Floyd's_tortoise_and_hare}
{Floyd's tortoise and hare algorithm}. I've previously implemented
this algorithm in C++ for a Codeforces contest:
\href{https://github.com/mathletedev/cp/blob/main/Codeforces/1137D.cpp}{GitHub}

Here is an algorithmic pseudocode for cycle detection:

\begin{minted}{c}
// to avoid null dereference
node *next(node *curr) {
  return curr == NULL ? NULL : curr->next;
}

bool cyclic(node *head) {
  node *hare = head, *tortoise = head;

  while (true) {
    hare = next(next(hare));
    tortoise = next(tortoise);

    // list terminates, thus is acyclic
    if (hare == NULL) {
      return false;
    }

    // hare and tortoise meet, thus is cyclic
    if (hare == tortoise) {
      return true;
    }
  }
}
\end{minted}

Since the hare travels at twice the speed of the tortoise, it
advances one extra node per iteration. Once the hare leaves the
starting node, the only way for the hare to encounter the tortoise
again is if there is a cycle in the linked list.

\newpage

\section{Problem 3}

(a), (b), (d), and (f) all apply.

(a) works with $c = 1$ and $n_0 = 20$.

(b) works with $c = 1$ and $n_0 = 1$.

(d) works with $c_1 = \frac{1}{200}$, $c_2 = 1$, and $n_0 = 20$.

(f) works because:

$\lim_{n \to \infty} \dfrac{\frac{n^3}{100} + 10n^2
+ n + 3}{n^2} = \lim_{n \to \infty} \dfrac{n^3}{n^2} = \lim_{n \to
\infty} n = \infty$.

\section{Problem 4}

$2 / N, 37, \sqrt N, N, N \log(\log N), \set{N \log N, N \log N^2}, N \log^2
N, \\ N^{1.5}, N^2, N^2 \log N, N^3, 2^{N / 2}, 2^N$

\section{Problem 5}

\subsection{}

\begin{quote}
  Prove that $5 \lg n = o(n \lg n)$.
\end{quote}

\begin{theorem}
  $5 \lg n = o(n \lg n)$
\end{theorem}

\begin{proof}
  First, we will show that $\lim_{n \to \infty} \dfrac{5 \lg n}{n \lg n} = 0$.

  $\lim_{n \to \infty} \dfrac{5 \lg n}{n \lg n} = \lim_{n \to
  \infty} \dfrac{5}{n} = 0.$

  Then $5 \lg n \ll n \lg n$ ($n \lg n$ grows faster than $5 \lg n$).

  Then
  $$ \exists c, n_0 \in \mathbb R, \forall n \in \mathbb R (n \geq
  n_0 \implies 5 \lg n < cn \lg n). $$

  Therefore, $5 \lg n = o(n \lg n)$.
\end{proof}

\newpage

\subsection{}

\begin{quote}
  Prove that $2 \lg n = O(\sqrt n)$.
\end{quote}

\begin{lemma}
  \label{growth}
  $\lg n \ll \sqrt n$.
\end{lemma}

\begin{proof}
  We will show that $\lim_{n \to \infty} \dfrac{\lg n}{\sqrt n} = 0$.

  \begin{align*}
    \lim_{n \to \infty} \dfrac{\lg n}{\sqrt n} & =
    \lim_{n \to \infty} \dfrac{\frac{d}{dn} \lg n}{\frac{d}{dn}
    \sqrt n} \quad \text{(By
    L'Hôpital's Rule)} \\
    & = \lim_{n \to \infty} \dfrac{\frac{1}{n \ln 2}}{\frac{1}{2 \sqrt n}} \\
    & = \lim_{n \to \infty} \dfrac{2 \sqrt n}{n \ln 2} \\
    & = \lim_{n \to \infty} \dfrac{2}{\ln(2) \sqrt n} \\
    & = 0.
  \end{align*}

  Therefore, $\lg n \ll \sqrt n$.
\end{proof}

\begin{theorem}
  $2 \lg n = O(\sqrt n)$.
\end{theorem}

\begin{proof}
  For $c = 2$, we have $2 \lg n \ll 2 \sqrt n$, which is equivalent to
  $\lg n \ll \sqrt n$ (\cref{growth}).

  For $n_0 = 1$, we have $2 \lg n = 0$ and $\sqrt n = 1$. $0 < 1$.

  Then, with $c = 2$ and $n_0 = 1$,
  $$ \forall n \in \mathbb R (n \geq 1 \implies 2 \lg n \leq
  2 \sqrt n). $$

  Therefore, $2 \lg n = O(\sqrt n)$.
\end{proof}

\newpage

\section{Problem 6}

\subsection{}

\begin{quote}
  Worst-case running time complexity: $\Theta (n^2)$.
\end{quote}

In this case, a swap is required at every index. We end up
constructing a "triangle" of operations, with $n - 1$ rows and $n - i$
operations per row. This would require $\sum_{i = 0}^{n - 1} (n - i)
= \sum_{i = 1}^{n} i = \frac{1}{2}n(n + 1) = \frac{n^2}{2} +
\frac{n}{2}$ operations.

With this polynomial, we take the most significant term to get a
worst-case running time complexity of $\Theta (n^2)$.

\subsection{}

\begin{quote}
  Best-case running time complexity: $\Theta (n)$.
\end{quote}

In this case, the algorithm scans the entire array only once,
determines that $\neg \exists x \in A (x < A_1)$, and halts.

The algorithm scans $n$ elements, so the best-case running time complexity is
$\Theta (n)$.

Note: If the algorithm is poorly designed such that it only searches
for $x = \min A$, not halting when $A$ is already sorted, then the
best-case running time complexity is $\Theta (n^2)$.

\end{document}
